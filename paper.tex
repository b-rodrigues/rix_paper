% Options for packages loaded elsewhere
\PassOptionsToPackage{unicode}{hyperref}
\PassOptionsToPackage{hyphens}{url}
\PassOptionsToPackage{dvipsnames,svgnames,x11names}{xcolor}
%
\documentclass[
  article]{jss}

\usepackage{amsmath,amssymb}
\usepackage{iftex}
\ifPDFTeX
  \usepackage[T1]{fontenc}
  \usepackage[utf8]{inputenc}
  \usepackage{textcomp} % provide euro and other symbols
\else % if luatex or xetex
  \usepackage{unicode-math}
  \defaultfontfeatures{Scale=MatchLowercase}
  \defaultfontfeatures[\rmfamily]{Ligatures=TeX,Scale=1}
\fi
\usepackage{lmodern}
\ifPDFTeX\else  
    % xetex/luatex font selection
\fi
% Use upquote if available, for straight quotes in verbatim environments
\IfFileExists{upquote.sty}{\usepackage{upquote}}{}
\IfFileExists{microtype.sty}{% use microtype if available
  \usepackage[]{microtype}
  \UseMicrotypeSet[protrusion]{basicmath} % disable protrusion for tt fonts
}{}
\makeatletter
\@ifundefined{KOMAClassName}{% if non-KOMA class
  \IfFileExists{parskip.sty}{%
    \usepackage{parskip}
  }{% else
    \setlength{\parindent}{0pt}
    \setlength{\parskip}{6pt plus 2pt minus 1pt}}
}{% if KOMA class
  \KOMAoptions{parskip=half}}
\makeatother
\usepackage{xcolor}
\setlength{\emergencystretch}{3em} % prevent overfull lines
\setcounter{secnumdepth}{-\maxdimen} % remove section numbering
% Make \paragraph and \subparagraph free-standing
\ifx\paragraph\undefined\else
  \let\oldparagraph\paragraph
  \renewcommand{\paragraph}[1]{\oldparagraph{#1}\mbox{}}
\fi
\ifx\subparagraph\undefined\else
  \let\oldsubparagraph\subparagraph
  \renewcommand{\subparagraph}[1]{\oldsubparagraph{#1}\mbox{}}
\fi


\providecommand{\tightlist}{%
  \setlength{\itemsep}{0pt}\setlength{\parskip}{0pt}}\usepackage{longtable,booktabs,array}
\usepackage{calc} % for calculating minipage widths
% Correct order of tables after \paragraph or \subparagraph
\usepackage{etoolbox}
\makeatletter
\patchcmd\longtable{\par}{\if@noskipsec\mbox{}\fi\par}{}{}
\makeatother
% Allow footnotes in longtable head/foot
\IfFileExists{footnotehyper.sty}{\usepackage{footnotehyper}}{\usepackage{footnote}}
\makesavenoteenv{longtable}
\usepackage{graphicx}
\makeatletter
\def\maxwidth{\ifdim\Gin@nat@width>\linewidth\linewidth\else\Gin@nat@width\fi}
\def\maxheight{\ifdim\Gin@nat@height>\textheight\textheight\else\Gin@nat@height\fi}
\makeatother
% Scale images if necessary, so that they will not overflow the page
% margins by default, and it is still possible to overwrite the defaults
% using explicit options in \includegraphics[width, height, ...]{}
\setkeys{Gin}{width=\maxwidth,height=\maxheight,keepaspectratio}
% Set default figure placement to htbp
\makeatletter
\def\fps@figure{htbp}
\makeatother

\usepackage{orcidlink,thumbpdf,lmodern}

\newcommand{\class}[1]{`\code{#1}'}
\newcommand{\fct}[1]{\code{#1()}}
\makeatletter
\@ifpackageloaded{tcolorbox}{}{\usepackage[skins,breakable]{tcolorbox}}
\@ifpackageloaded{fontawesome5}{}{\usepackage{fontawesome5}}
\definecolor{quarto-callout-color}{HTML}{909090}
\definecolor{quarto-callout-note-color}{HTML}{0758E5}
\definecolor{quarto-callout-important-color}{HTML}{CC1914}
\definecolor{quarto-callout-warning-color}{HTML}{EB9113}
\definecolor{quarto-callout-tip-color}{HTML}{00A047}
\definecolor{quarto-callout-caution-color}{HTML}{FC5300}
\definecolor{quarto-callout-color-frame}{HTML}{acacac}
\definecolor{quarto-callout-note-color-frame}{HTML}{4582ec}
\definecolor{quarto-callout-important-color-frame}{HTML}{d9534f}
\definecolor{quarto-callout-warning-color-frame}{HTML}{f0ad4e}
\definecolor{quarto-callout-tip-color-frame}{HTML}{02b875}
\definecolor{quarto-callout-caution-color-frame}{HTML}{fd7e14}
\makeatother
\makeatletter
\@ifpackageloaded{caption}{}{\usepackage{caption}}
\AtBeginDocument{%
\ifdefined\contentsname
  \renewcommand*\contentsname{Table of contents}
\else
  \newcommand\contentsname{Table of contents}
\fi
\ifdefined\listfigurename
  \renewcommand*\listfigurename{List of Figures}
\else
  \newcommand\listfigurename{List of Figures}
\fi
\ifdefined\listtablename
  \renewcommand*\listtablename{List of Tables}
\else
  \newcommand\listtablename{List of Tables}
\fi
\ifdefined\figurename
  \renewcommand*\figurename{Figure}
\else
  \newcommand\figurename{Figure}
\fi
\ifdefined\tablename
  \renewcommand*\tablename{Table}
\else
  \newcommand\tablename{Table}
\fi
}
\@ifpackageloaded{float}{}{\usepackage{float}}
\floatstyle{ruled}
\@ifundefined{c@chapter}{\newfloat{codelisting}{h}{lop}}{\newfloat{codelisting}{h}{lop}[chapter]}
\floatname{codelisting}{Listing}
\newcommand*\listoflistings{\listof{codelisting}{List of Listings}}
\makeatother
\makeatletter
\makeatother
\makeatletter
\@ifpackageloaded{caption}{}{\usepackage{caption}}
\@ifpackageloaded{subcaption}{}{\usepackage{subcaption}}
\makeatother
\makeatletter
\@ifpackageloaded{tcolorbox}{}{\usepackage[skins,breakable]{tcolorbox}}
\makeatother
\makeatletter
\@ifundefined{shadecolor}{\definecolor{shadecolor}{rgb}{.97, .97, .97}}{}
\makeatother
\makeatletter
\makeatother
\makeatletter
\ifdefined\Shaded\renewenvironment{Shaded}{\begin{tcolorbox}[sharp corners, borderline west={3pt}{0pt}{shadecolor}, frame hidden, breakable, boxrule=0pt, interior hidden, enhanced]}{\end{tcolorbox}}\fi
\makeatother
\ifLuaTeX
  \usepackage{selnolig}  % disable illegal ligatures
\fi
\IfFileExists{bookmark.sty}{\usepackage{bookmark}}{\usepackage{hyperref}}
\IfFileExists{xurl.sty}{\usepackage{xurl}}{} % add URL line breaks if available
\urlstyle{same} % disable monospaced font for URLs
\hypersetup{
  pdftitle={Reproducible development environments with rix},
  pdfauthor={Bruno Rodrigues; Philipp Baumann},
  pdfkeywords={reproducibility, R, Nix},
  colorlinks=true,
  linkcolor={blue},
  filecolor={Maroon},
  citecolor={Blue},
  urlcolor={Blue},
  pdfcreator={LaTeX via pandoc}}

%% -- Article metainformation (author, title, ...) -----------------------------

%% Author information
\author{Bruno Rodrigues~\orcidlink{0000-0002-3211-3689}\\Ministry of
Research and Higher education, Luxembourg \And Philipp Baumann\\Plus
Affiliation}
\Plainauthor{Bruno Rodrigues, Philipp Baumann} %% comma-separated

\title{Reproducible development environments with rix}
\Plaintitle{Reproducible development environments with
rix} %% without formatting

%% an abstract and keywords
\Abstract{In order to create an analysis that is easily reproducible, it
is not enough to write clean code and document it well. One must also
make sure to list all the dependencies of the analysis clearly and
ideally provide an easy way to install said dependencies. There are
several tools that can be used to list dependencies and to make them
easily installable by someone that wishes to reproduce a study, such as
\proglang{Docker}, a containerization solution. This paper will present
the Nix package manager, and an \proglang{R} package called \pkg{rix}
that lowers Nix's learning curve for users of the \proglang{R}
programming language.}

%% at least one keyword must be supplied
\Keywords{reproducibility, \proglang{R}, \proglang{Nix}}
\Plainkeywords{reproducibility, R, Nix}

%% publication information
%% NOTE: Typically, this can be left commented and will be filled out by the technical editor
%% \Volume{50}
%% \Issue{9}
%% \Month{June}
%% \Year{2012}
%% \Submitdate{2012-06-04}
%% \Acceptdate{2012-06-04}
%% \setcounter{page}{1}
%% \Pages{1--xx}

%% The address of (at least) one author should be given
%% in the following format:
\Address{
Bruno Rodrigues\\
Department of Statistics\\
18, Montée de la Pétrusse\\
Luxembourg Luxembourg\\
E-mail: \email{bruno@brodrigues.co}\\
URL: \url{https://www.brodrigues.co}\\
\\~
Philipp Baumann\\
\\~

}

\begin{document}
\maketitle
\section{Introduction: Reproducibility is also about
software}\label{sec-intro}

\citet{peng2011} introduced the idea of reproducibility being on a
continuum: on one of the ends of this continuum, we only have access to
the paper describing the studies, which is not reproducible at all.
Then, if in addition, to this paper we make the original source code of
the analysis that was written to compute the results of the study
available, reproducibility is improved, albeit only by a little. Adding
the original data improves reproducibility yet again. Finally, if to all
this we add what Roger Peng named the \emph{linked and executable code
and data}, we reach the gold standard of full replication.

What is this \emph{linked and executable code and data}? Another way to
name this crucial piece of the reproducibility puzzle is
\emph{computational environment}. The computational environment is all
the software required to actually run the analysis. Here too, we can
speak of a continuum. One could simply name and list the software used:
for example, the \proglang{R} programming language. Sometimes, authors
have the courtesy to also state the version of \proglang{R} used. Some
authors go further, and also list the packages used, and ideally with
their versions as well. Authors rarely state the operating system on
which the analysis was done, even though it has been shown that running
the same analysis with the same software but on different operating
systems could lead to different results, as described in
\citet{neupane2019}. Authors also only very rarely provide instructions
to install the required tools and software in order to reproduce their
studies.

Tools can be used at each of these steps to reach the gold standard of
full replication. Let's first consider the task of listing the software
used. \proglang{R} provides the \texttt{sessionInfo()} function whose
output can be saved into a file. Below is an example output of
\texttt{sessionInfo()}:

\begin{verbatim}
sessionInfo()
\end{verbatim}

\begin{verbatim}
R version 4.3.2 (2023-10-31)
Platform: aarch64-unknown-linux-gnu (64-bit)
Running under: Ubuntu 22.04.3 LTS

Matrix products: default
BLAS:   /usr/lib/aarch64-linux-gnu/openblas-pthread/libblas.so.3 
LAPACK: /usr/lib/aarch64-linux-gnu/openblas-pthread/libopenblasp[...]

locale:
 [1] LC_CTYPE=en_US.UTF-8       LC_NUMERIC=C              
 [3] LC_TIME=en_US.UTF-8        LC_COLLATE=en_US.UTF-8    
 [5] LC_MONETARY=en_US.UTF-8    LC_MESSAGES=en_US.UTF-8   
 [7] LC_PAPER=en_US.UTF-8       LC_NAME=C                 
 [9] LC_ADDRESS=C               LC_TELEPHONE=C            
[11] LC_MEASUREMENT=en_US.UTF-8 LC_IDENTIFICATION=C       

time zone: Etc/UTC
tzcode source: system (glibc)

attached base packages:
[1] stats     graphics  grDevices utils     datasets  methods  
[7] base     

other attached packages:
[1] nnet_7.3-19  mgcv_1.9-0   nlme_3.1-163

loaded via a namespace (and not attached):
[1] compiler_4.3.2 Matrix_1.6-1.1 tools_4.3.2    splines_4.3.2 
[5] grid_4.3.2     lattice_0.21-9
\end{verbatim}

If an author provides this information, other people trying to reproduce
the study (or the author him- or herself in the future) can read this
file and see which version of \proglang{R} was used, and which packages
(and their versions) were used as well. However, others would still need
to install the correct software themselves. An alternative to this is
instead to use the \pkg{renv} package which generates a so-called
\texttt{renv.lock} file which also lists \proglang{R} and package
versions. Here is an example of such an \texttt{renv.lock} file:

\begin{verbatim}
{
"R": {
  "Version": "4.2.2",
  "Repositories": [
  {
   "Name": "CRAN",
   "URL": "https://packagemanager.rstudio.com/all/latest"
  }
  ]
},
"Packages": {
  "MASS": {
    "Package": "MASS",
    "Version": "7.3-58.1",
    "Source": "Repository",
    "Repository": "CRAN",
    "Hash": "762e1804143a332333c054759f89a706",
    "Requirements": []
  },
  "Matrix": {
    "Package": "Matrix",
    "Version": "1.5-1",
    "Source": "Repository",
    "Repository": "CRAN",
    "Hash": "539dc0c0c05636812f1080f473d2c177",
    "Requirements": [
      "lattice"
    ]

    ***and many more packages***
\end{verbatim}

This file lists every package alongside their versions and the
repository from which they were downloaded. Generating this file only
requires one to run the \texttt{renv::init()} function. Someone else can
then restore the same package library by running
\texttt{renv::restore()}. The exact same packages get installed in an
isolated, project-specific, library which doesn't interfere with the
other, main, library of the user. \pkg{renv} does not restore the
\proglang{R} itself though, so installing the right version of R needs
to be handled separately. Before continuing, it should be noted that
other packages exist which provide similar functionality to \pkg{renv}:
there is \pkg{groundhog} by \citet{simonsohn2023} which makes it rather
easy to install packages as they were on CRAN at a given date. For
example, the code snippet below install the \pkg{purrr} and
\pkg{ggplot2} packages as they were on April 4th, 2017:

\begin{verbatim}
groundhog.library("
    library(purrr)
    library(ggplot2)",
    "2017-10-04",
    tolerate.R.version = "4.2.2")
\end{verbatim}

These packages also get installed in a project-specfic library so there
is no interferences between these packages and other versions of the
same packages that one might use for other projects. Because
\pkg{groundhog} does not install \proglang{R} itself, users should
either install the required version themselves, or they should use the
\texttt{tolerate.R.version} argument as shown in the example above.
Otherwise, \pkg{groundhog} would not continue with the installation of
the packages. Another such package, developed by \citet{chan2023} is
\pkg{rang}, which also installs packages as they were on a given date.
Yet another way to install packages as they were on a give date is to
use the Posit Package Manager, which provides snapshots of CRAN. For
example, to install the required packages for an analysis as they were
on the 30th of June 2023, one could add the following line to the
\texttt{.Rprofile} file:

\begin{verbatim}
options(repos = c(REPO_NAME = "https://packagemanager.posit.co/cran/
                               __linux__/jammy/2023-06-30"))
\end{verbatim}

The \texttt{.Rprofile} file gets read by R when starting a new session,
which means that every call to the \texttt{install.packages()} function
will now install the packages from this snapshotted mirror.

The next step in reaching the gold standard of reproducibility would be
to not only install the right packages used for the analysis, but also
the right version of \proglang{R}. Of course, it would be possible to
install the right version manually, but here too, there are tools that
simplify the process such as \proglang{rig} by the R infrastructure team
\citeyearpar{rlib2023}. One could thus install the right version of
\proglang{R} using \proglang{rig} allows to easily install different
versions of \proglang{R}, one could use to install the right version to
reproduce an analysis, and then use one of the listed packages above to
install the right library of of \proglang{R} packages. However, this
involves many manual steps and is thus error prone. It is also
time-consuming to do so, especially if one wants to reproduce many
studies.

The final step towards the gold standard of reproducibility is to
package the right version of \proglang{R} and \proglang{R} packages
inside a \proglang{Docker} image. \proglang{Docker} is a
containerisation tool: using \proglang{Docker} it is possible to package
some \emph{data product} with its dependencies into an image. From this
image, containers can be run to reproduce the data product with its
exact dependencies. A statistical analysis, from the simplest to the
most complex one, can be seen as such a data product that comes with
many software dependencies. \emph{Dockerizing} an analysis consists in
first building an image: in the build step, the dependencies of the
analysis have to be installed and this can be achieved using the tools
mentioned above. The scripts to run the analysis and the data are also
added to the image at build time. Instances of that image, called
containers, can then be executed, which give access to the exact
development environment originally used for the study. It is also
possible to script the building of the analysis itself, such that in one
single call to the \texttt{docker\ run} command, a complete analysis is
reproduced.

The Rocker project initiated by \citet{boettiger2017} provides many
pre-built \proglang{Docker} to the \proglang{R} community of users.
These images can be used as bases to build other images containing
statistical analyses more easily than starting from a bare-bones image.

An optional step towards the gold standard is to use a build automation
tool such as \proglang{Make} to run the whole analysis when the
container is executed. Build automation tools make it easier to run
arbitrary code in a series of well-defined steps. \proglang{R}
programmers can use the \pkg{targets} by \citet{landau2021} as a build
automation tool.

\citet{mcdermott2021} is an example of a scientific study that reached
the gold standard of reproducibility. The author of this paper set up an
accompagnying Github repository to the paper\footnote{https://github.com/grantmcdermott/skeptic-priors}
containing all the instructions to install the required software and
then run the analysis. If we take a closer look at this repository, we
will notice that many of the tools previously mentioned were used to
capture the compatutational environment and make it available to other
researchers:

\begin{itemize}
\tightlist
\item
  Packages and their versions were listed and saved into an
  \texttt{renv.lock} file;
\item
  A \texttt{Makefile} was used to run the whole analysis and compile the
  paper;
\item
  A \texttt{Dockerfile} was used to provide the complete computational
  environment, including the right version of \proglang{R} and run the
  whole analysis easily.
\end{itemize}

However, reaching this gold standard is quite costly: one needs to learn
a tool to deal with package versions, then \proglang{Docker} for all the
other software, including the programming language that was used.
Ideally, a build automation tool such as \proglang{make} should be added
to the list. It should also be noted that these requirements are not
limited to the \proglang{R} programming language. A very similar
approach should be taken if one uses \proglang{Python} for statistical
analysis instead.

As an alternative to the above approach, I will present the
\proglang{Nix} package manager, which available for all major operating
system and focuses on installing and building software in a reproducible
manner. To make \proglang{Nix} more accessible to \proglang{R}
programmers, I developed the \pkg{rix} which I will also present in this
article.

\section{The Nix package manager}\label{the-nix-package-manager}

\proglang{Nix} is a package manager that can be used to install and
build software in a completely reproducible manner. As of writing, it
contains more than 80.000 packages, and the entirety of CRAN and
Bioconductor is available through \proglang{Nix}'s repositories. This
means that using \proglang{Nix}, it is possible to install not only
\proglang{R}, but also all the packages required for a project. The
reason why one should use \proglang{Nix} to install \proglang{R}
packages, and not use the usual, built-in, \texttt{install.packages()}
\proglang{R} function instead, is that when installing a package with
\proglang{Nix}, \proglang{Nix} makes sure to install every dependency of
every package, whether this dependency is another \proglang{R} package
or a system-level dependency.

For example, the \pkg{xlsx} \proglang{R} package requires the
\proglang{Java} programming language to be installed to successfully
install. Depending on the what system one tries to install \pkg{xlsx},
installing \proglang{Java} might not easy, nor even possible. But with
\proglang{Nix}, it suffices to declare that the \pkg{xlsx} package is
needed for the project, and \proglang{Nix} figures out automatically
that \proglang{Java} is required and installs and configures it. It all
just happens without any required intervention from the user.

But where do these packages come from? When installing a package using
\proglang{Nix}, an expression written in the \proglang{Nix} programming
language gets downloaded from the \texttt{nixpkgs} Github repository and
evaluated. This expression contains a so-called \emph{derivation}. A
derivation defines a build: its dependencies, commands to build and
install the package in question, and then an output. Most of the time, a
derivation downloads source code, builds the software from the source
and then outputs a compiled binary. Derivations are extremely flexible,
and by writing one's own, it is possible to define and build environment
for a project in a reproducible manner.

Because the whole set of \proglang{Nix} expression is hosted on Github,
it is possible to pin a specific revision of \texttt{nixpkgs} to ensure
reproducibility of our project. Pinning a revision ensures that every
package that Nix installs will always be at exactly the same versions,
regardless of when in the future the packages get installed.

With Nix, it is essentially possible to replace \pkg{renv} and Docker
combined, or in the case of \proglang{Python}, replace
\texttt{requirements.txt} files which list package dependencies for
\proglang{Python} projects. It is also possible to build multi-language
environments, containing R and Python, a LaTeX distribution, and even
install one's favourite text editor to edit the project's source code.

\section*{Acknowledgments}\label{acknowledgments}
\addcontentsline{toc}{section}{Acknowledgments}

\begin{tcolorbox}[enhanced jigsaw, arc=.35mm, left=2mm, toprule=.15mm, colback=white, leftrule=.75mm, breakable, rightrule=.15mm, bottomrule=.15mm, opacityback=0]

All acknowledgments (note the AE spelling) should be collected in this
unnumbered section before the references. It may contain the usual
information about funding and feedback from colleagues/reviewers/etc.
Furthermore, information such as relative contributions of the authors
may be added here (if any).

\end{tcolorbox}

\section*{References}\label{references}
\addcontentsline{toc}{section}{References}

\renewcommand{\bibsection}{}
\bibliography{bibliography.bib}

\newpage{}

\section*{More technical details}\label{sec-techdetails}
\addcontentsline{toc}{section}{More technical details}

\begin{tcolorbox}[enhanced jigsaw, arc=.35mm, left=2mm, toprule=.15mm, colback=white, leftrule=.75mm, breakable, rightrule=.15mm, bottomrule=.15mm, opacityback=0]

Appendices can be included after the bibliography (with a page break).
Each section within the appendix should have a proper section title
(rather than just \emph{Appendix}).

For more technical style details, please check out JSS's style FAQ at
{[}https://www.jstatsoft.org/pages/view/style\#frequently-asked-questions{]}
which includes the following topics:

\begin{itemize}
\tightlist
\item
  Title vs.~sentence case.
\item
  Graphics formatting.
\item
  Naming conventions.
\item
  Turning JSS manuscripts into \proglang{R} package vignettes.
\item
  Trouble shooting.
\item
  Many other potentially helpful details\ldots{}
\end{itemize}

\end{tcolorbox}

\section*{Using BibTeX}\label{sec-bibtex}
\addcontentsline{toc}{section}{Using BibTeX}

\begin{tcolorbox}[enhanced jigsaw, arc=.35mm, left=2mm, toprule=.15mm, colback=white, leftrule=.75mm, breakable, rightrule=.15mm, bottomrule=.15mm, opacityback=0]

References need to be provided in a \textsc{Bib}{\TeX} file
(\texttt{.bib}). All references should be made with \texttt{@cite}
syntax. This commands yield different formats of author-year citations
and allow to include additional details (e.g.,pages, chapters, \dots) in
brackets. In case you are not familiar with these commands see the JSS
style FAQ for details.

Cleaning up \textsc{Bib}{\TeX} files is a somewhat tedious task --
especially when acquiring the entries automatically from mixed online
sources. However, it is important that informations are complete and
presented in a consistent style to avoid confusions. JSS requires the
following format.

\begin{itemize}
\tightlist
\item
  item JSS-specific markup (\texttt{\textbackslash{}proglang},
  \texttt{\textbackslash{}pkg}, \texttt{\textbackslash{}code}) should be
  used in the references.
\item
  item Titles should be in title case.
\item
  item Journal titles should not be abbreviated and in title case.
\item
  item DOIs should be included where available.
\item
  item Software should be properly cited as well. For \proglang{R}
  packages \texttt{citation("pkgname")} typically provides a good
  starting point.
\end{itemize}

\end{tcolorbox}




\end{document}
