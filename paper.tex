% Options for packages loaded elsewhere
\PassOptionsToPackage{unicode}{hyperref}
\PassOptionsToPackage{hyphens}{url}
\PassOptionsToPackage{dvipsnames,svgnames,x11names}{xcolor}
%
\documentclass[
  article]{jss}

\usepackage{amsmath,amssymb}
\usepackage{iftex}
\ifPDFTeX
  \usepackage[T1]{fontenc}
  \usepackage[utf8]{inputenc}
  \usepackage{textcomp} % provide euro and other symbols
\else % if luatex or xetex
  \usepackage{unicode-math}
  \defaultfontfeatures{Scale=MatchLowercase}
  \defaultfontfeatures[\rmfamily]{Ligatures=TeX,Scale=1}
\fi
\usepackage{lmodern}
\ifPDFTeX\else  
    % xetex/luatex font selection
\fi
% Use upquote if available, for straight quotes in verbatim environments
\IfFileExists{upquote.sty}{\usepackage{upquote}}{}
\IfFileExists{microtype.sty}{% use microtype if available
  \usepackage[]{microtype}
  \UseMicrotypeSet[protrusion]{basicmath} % disable protrusion for tt fonts
}{}
\makeatletter
\@ifundefined{KOMAClassName}{% if non-KOMA class
  \IfFileExists{parskip.sty}{%
    \usepackage{parskip}
  }{% else
    \setlength{\parindent}{0pt}
    \setlength{\parskip}{6pt plus 2pt minus 1pt}}
}{% if KOMA class
  \KOMAoptions{parskip=half}}
\makeatother
\usepackage{xcolor}
\setlength{\emergencystretch}{3em} % prevent overfull lines
\setcounter{secnumdepth}{-\maxdimen} % remove section numbering
% Make \paragraph and \subparagraph free-standing
\ifx\paragraph\undefined\else
  \let\oldparagraph\paragraph
  \renewcommand{\paragraph}[1]{\oldparagraph{#1}\mbox{}}
\fi
\ifx\subparagraph\undefined\else
  \let\oldsubparagraph\subparagraph
  \renewcommand{\subparagraph}[1]{\oldsubparagraph{#1}\mbox{}}
\fi


\providecommand{\tightlist}{%
  \setlength{\itemsep}{0pt}\setlength{\parskip}{0pt}}\usepackage{longtable,booktabs,array}
\usepackage{calc} % for calculating minipage widths
% Correct order of tables after \paragraph or \subparagraph
\usepackage{etoolbox}
\makeatletter
\patchcmd\longtable{\par}{\if@noskipsec\mbox{}\fi\par}{}{}
\makeatother
% Allow footnotes in longtable head/foot
\IfFileExists{footnotehyper.sty}{\usepackage{footnotehyper}}{\usepackage{footnote}}
\makesavenoteenv{longtable}
\usepackage{graphicx}
\makeatletter
\def\maxwidth{\ifdim\Gin@nat@width>\linewidth\linewidth\else\Gin@nat@width\fi}
\def\maxheight{\ifdim\Gin@nat@height>\textheight\textheight\else\Gin@nat@height\fi}
\makeatother
% Scale images if necessary, so that they will not overflow the page
% margins by default, and it is still possible to overwrite the defaults
% using explicit options in \includegraphics[width, height, ...]{}
\setkeys{Gin}{width=\maxwidth,height=\maxheight,keepaspectratio}
% Set default figure placement to htbp
\makeatletter
\def\fps@figure{htbp}
\makeatother

\usepackage{orcidlink,thumbpdf,lmodern}

\newcommand{\class}[1]{`\code{#1}'}
\newcommand{\fct}[1]{\code{#1()}}
\makeatletter
\@ifpackageloaded{tcolorbox}{}{\usepackage[skins,breakable]{tcolorbox}}
\@ifpackageloaded{fontawesome5}{}{\usepackage{fontawesome5}}
\definecolor{quarto-callout-color}{HTML}{909090}
\definecolor{quarto-callout-note-color}{HTML}{0758E5}
\definecolor{quarto-callout-important-color}{HTML}{CC1914}
\definecolor{quarto-callout-warning-color}{HTML}{EB9113}
\definecolor{quarto-callout-tip-color}{HTML}{00A047}
\definecolor{quarto-callout-caution-color}{HTML}{FC5300}
\definecolor{quarto-callout-color-frame}{HTML}{acacac}
\definecolor{quarto-callout-note-color-frame}{HTML}{4582ec}
\definecolor{quarto-callout-important-color-frame}{HTML}{d9534f}
\definecolor{quarto-callout-warning-color-frame}{HTML}{f0ad4e}
\definecolor{quarto-callout-tip-color-frame}{HTML}{02b875}
\definecolor{quarto-callout-caution-color-frame}{HTML}{fd7e14}
\makeatother
\makeatletter
\@ifpackageloaded{caption}{}{\usepackage{caption}}
\AtBeginDocument{%
\ifdefined\contentsname
  \renewcommand*\contentsname{Table of contents}
\else
  \newcommand\contentsname{Table of contents}
\fi
\ifdefined\listfigurename
  \renewcommand*\listfigurename{List of Figures}
\else
  \newcommand\listfigurename{List of Figures}
\fi
\ifdefined\listtablename
  \renewcommand*\listtablename{List of Tables}
\else
  \newcommand\listtablename{List of Tables}
\fi
\ifdefined\figurename
  \renewcommand*\figurename{Figure}
\else
  \newcommand\figurename{Figure}
\fi
\ifdefined\tablename
  \renewcommand*\tablename{Table}
\else
  \newcommand\tablename{Table}
\fi
}
\@ifpackageloaded{float}{}{\usepackage{float}}
\floatstyle{ruled}
\@ifundefined{c@chapter}{\newfloat{codelisting}{h}{lop}}{\newfloat{codelisting}{h}{lop}[chapter]}
\floatname{codelisting}{Listing}
\newcommand*\listoflistings{\listof{codelisting}{List of Listings}}
\makeatother
\makeatletter
\makeatother
\makeatletter
\@ifpackageloaded{caption}{}{\usepackage{caption}}
\@ifpackageloaded{subcaption}{}{\usepackage{subcaption}}
\makeatother
\makeatletter
\@ifpackageloaded{tcolorbox}{}{\usepackage[skins,breakable]{tcolorbox}}
\makeatother
\makeatletter
\@ifundefined{shadecolor}{\definecolor{shadecolor}{rgb}{.97, .97, .97}}{}
\makeatother
\makeatletter
\makeatother
\makeatletter
\ifdefined\Shaded\renewenvironment{Shaded}{\begin{tcolorbox}[sharp corners, breakable, enhanced, borderline west={3pt}{0pt}{shadecolor}, frame hidden, interior hidden, boxrule=0pt]}{\end{tcolorbox}}\fi
\makeatother
\ifLuaTeX
  \usepackage{selnolig}  % disable illegal ligatures
\fi
\IfFileExists{bookmark.sty}{\usepackage{bookmark}}{\usepackage{hyperref}}
\IfFileExists{xurl.sty}{\usepackage{xurl}}{} % add URL line breaks if available
\urlstyle{same} % disable monospaced font for URLs
\hypersetup{
  pdftitle={Reproducible development environments with rix},
  pdfauthor={Bruno Rodrigues; Philipp Baumann},
  pdfkeywords={reproducibility, R, Nix},
  colorlinks=true,
  linkcolor={blue},
  filecolor={Maroon},
  citecolor={Blue},
  urlcolor={Blue},
  pdfcreator={LaTeX via pandoc}}

%% -- Article metainformation (author, title, ...) -----------------------------

%% Author information
\author{Bruno Rodrigues~\orcidlink{0000-0002-3211-3689}\\Ministry of
Research and Higher education, Luxembourg \And Philipp Baumann\\Plus
Affiliation}
\Plainauthor{Bruno Rodrigues, Philipp Baumann} %% comma-separated

\title{Reproducible development environments with rix}
\Plaintitle{Reproducible development environments with
rix} %% without formatting

%% an abstract and keywords
\Abstract{In order to create an analysis that is easily reproducible, it
is not enough to write clean code and document it well. One must also
make sure to list all the dependencies of the analysis clearly and
ideally provide an easy way to install said dependencies. There are
several tools that can be used to list dependencies and to make them
easily installable by someone that wishes to reproduce a study, such as
Docker, a containerization solution. This paper will present the Nix
package manager, and an R package called \{rix\} that lowers Nix's
learning curve for users of the R programming language.}

%% at least one keyword must be supplied
\Keywords{reproducibility, \proglang{R}, \proglang{Nix}}
\Plainkeywords{reproducibility, R, Nix}

%% publication information
%% NOTE: Typically, this can be left commented and will be filled out by the technical editor
%% \Volume{50}
%% \Issue{9}
%% \Month{June}
%% \Year{2012}
%% \Submitdate{2012-06-04}
%% \Acceptdate{2012-06-04}
%% \setcounter{page}{1}
%% \Pages{1--xx}

%% The address of (at least) one author should be given
%% in the following format:
\Address{
Bruno Rodrigues\\
Department of Statistics\\
18, Montée de la Pétrusse\\
Luxembourg Luxembourg\\
E-mail: \email{bruno@brodrigues.co}\\
URL: \url{https://www.brodrigues.co}\\
\\~
Philipp Baumann\\
\\~

}

\begin{document}
\maketitle
\section{Introduction: Reproducibility is also about
software}\label{sec-intro}

\begin{tcolorbox}[enhanced jigsaw, breakable, left=2mm, leftrule=.75mm, opacityback=0, arc=.35mm, rightrule=.15mm, bottomrule=.15mm, colback=white, toprule=.15mm]

The introduction is in principle ``as usual''. However, it should
usually embed both the implemented \emph{methods} and the
\emph{software} into the respective relevant literature. For the latter
both competing and complementary software should be discussed (within
the same software environment and beyond), bringing out relative
(dis)advantages. All software mentioned should be properly
\texttt{@cited}'d.~(See also \hyperref[sec-bibtex]{Using BibTeX} for
more details on \textsc{Bib}{\TeX}.)

For writing about software JSS requires authors to use the markup
\texttt{{[}{]}\{.proglang\}} (programming languages and large
programmable systems), \texttt{{[}{]}\{.pkg\}} (software packages), back
ticks like `code` for code (functions, commands, arguments, etc.).

If there is such markup in (sub)section titles (as above), a plain text
version has to be provided in the {\LaTeX} command as well. Below we
also illustrate how abbrevations should be introduced and citation
commands can be employed. See the {\LaTeX} code for more details.

\end{tcolorbox}

\citet{peng2011} introduced the idea of reproducibility being on a
continuum: on one of the ends of this continuum, we only have access to
the paper describing the studies, which is not reproducible at all.
Then, if in addition, to this paper we make the original source code of
the analysis that was written to compute the results of the study
available, reproducibility is improved, albeit only by a little. Adding
the original data improves reproducibility yet again. Finally, if to all
this we add what Roger Peng named the \emph{linked and executable code
and data}, we reach the gold standard of full replication.

What is this \emph{linked and executable code and data}? Another way to
name this crucial piece of the reproducibility puzzle is
\emph{computational environment}. The computational environment is all
the software required to actually run the analysis. Here too, we can
speak of a continuum. One could simply name and list the software used:
for example, the \proglang{R} programming language. Sometimes, authors
have the courtesy to also state the version of \proglang{R} used. Some
authors go further, and also list the packages used, and ideally with
their versions as well. Authors rarely state the operating system on
which the analysis was done, even though it has been shown that running
the same analysis with the same software but on different operating
systems could lead to different results, as described in
\citet{neupane2019}. Authors also only very rarely provide instructions
to install the required tools and software in order to reproduce their
studies.

There are exceptions of course, a great example of a paper that provides
everything needed to reproduce its results is \citet{mcdermott2021}. The
author of this paper set up an accompagnying Github repository to the
paper\footnote{https://github.com/grantmcdermott/skeptic-priors}
containing all the instructions to install the required software and
then run the analysis. If we take a closer look at this repository, we
will notice that several tools were used to capture the compatutational
environment and make it available to other researchers:

\begin{itemize}
\tightlist
\item
  The version of \proglang{R} was stated;
\item
  Packages and their versions were listed and saved into an
  \texttt{renv.lock} file;
\item
  A \texttt{Makefile} was used to run the whole analysis and compile the
  paper;
\item
  A \texttt{Dockerfile} was used to provide the complete computational
  environment, including the right version of \proglang{R} and run the
  whole analysis easily.
\end{itemize}

As we go down this list, we get closer to the gold standard of a
perfectly reproducible study.

However, reaching this gold standard is quite costly: one needs to learn
a tool to deal with package versions, then a build automation tool such
as \proglang{make} and ideally \proglang{Docker} should be added to the
list. \proglang{Docker} is a tool that makes it possible to run
arbitrary code in a completely controlled and isolated environment as a
way to capture the complete underlying system libraries.

\subsection{Prior art}\label{subsec-prior}

\proglang{R} provides a very flexible implementation of the general GLM
framework in the function \fct{glm} \citet{ChambersHastie1992} in the
\pkg{stats} package. Its most important arguments are

\begin{verbatim}
glm(formula, data, subset, na.action, weights, offset,
  family = gaussian, start = NULL, control = glm.control(…),
  model = TRUE, y = TRUE, x = FALSE, …)
\end{verbatim}

where \texttt{formula} plus \texttt{data} is the now standard way of
specifying regression relationships in \proglang{R}/\proglang{S}
introduced in \citet{ChambersHastie1992}. The remaining arguments in the
first line (\texttt{subset}, \texttt{na.action}, \texttt{weights}, and
\texttt{offset}) are also standard for setting up formula-based
regression models in \proglang{R}/\proglang{S}. The arguments in the
second line control aspects specific to GLMs while the arguments in the
last line specify which components are returned in the fitted model
object (of class \class{glm} which inherits from \class{lm}). For
further arguments to \fct{glm} (including alternative specifications of
starting values) see \texttt{?glm}. For estimating a Poisson model
\texttt{family\ =\ poisson} has to be specified.

\begin{tcolorbox}[enhanced jigsaw, breakable, left=2mm, leftrule=.75mm, opacityback=0, arc=.35mm, rightrule=.15mm, bottomrule=.15mm, colback=white, toprule=.15mm]

As the synopsis above is a code listing that is not meant to be
executed, one can use either the dedicated \texttt{\{Code\}} environment
or a simple \texttt{\{verbatim\}} environment for this. Again, spaces
before and after should be avoided.

Finally, there might be a reference to a \texttt{\{table\}} such as
Table~\ref{tbl-overview}. Usually, these are placed at the top of the
page (\texttt{{[}t!{]}}), centered (\texttt{\textbackslash{}centering}),
with a caption below the table, column headers and captions in sentence
style, and if possible avoiding vertical lines.

\end{tcolorbox}

\begin{longtable}[]{@{}
  >{\raggedright\arraybackslash}p{(\columnwidth - 6\tabcolsep) * \real{0.1216}}
  >{\raggedright\arraybackslash}p{(\columnwidth - 6\tabcolsep) * \real{0.1216}}
  >{\raggedright\arraybackslash}p{(\columnwidth - 6\tabcolsep) * \real{0.1216}}
  >{\raggedright\arraybackslash}p{(\columnwidth - 6\tabcolsep) * \real{0.6351}}@{}}
\toprule\noalign{}
\begin{minipage}[b]{\linewidth}\raggedright
Type
\end{minipage} & \begin{minipage}[b]{\linewidth}\raggedright
Distribution
\end{minipage} & \begin{minipage}[b]{\linewidth}\raggedright
Method
\end{minipage} & \begin{minipage}[b]{\linewidth}\raggedright
Description
\end{minipage} \\
\midrule\noalign{}
\endfirsthead
\toprule\noalign{}
\begin{minipage}[b]{\linewidth}\raggedright
Type
\end{minipage} & \begin{minipage}[b]{\linewidth}\raggedright
Distribution
\end{minipage} & \begin{minipage}[b]{\linewidth}\raggedright
Method
\end{minipage} & \begin{minipage}[b]{\linewidth}\raggedright
Description
\end{minipage} \\
\midrule\noalign{}
\endhead
\bottomrule\noalign{}
\endlastfoot
GLM & Poisson & ML & Poisson regression: classical GLM, estimated by
maximum likelihood (ML) \\
& & Quasi & ``Quasi-Poisson regression'\,': same mean function,
estimated by quasi-ML (QML) or equivalently generalized estimating
equations (GEE), inference adjustment via estimated dispersion
parameter \\
& & Adjusted & ``Adjusted Poisson regression'\,': same mean function,
estimated by QML/GEE, inference adjustment via sandwich covariances \\
& NB & ML & NB regression: extended GLM, estimated by ML including
additional shape parameter \\
Zero-augmented & Poisson & ML & Zero-inflated Poisson (ZIP), hurdle
Poisson \\
& NB & ML & Zero-inflated NB (ZINB), hurdle NB \\
\caption{Overview of various count regression models. The table is
usually placed at the top of the page (\texttt{{[}t!{]}}), centered
(\texttt{centering}), has a caption below the table, column headers and
captions are in sentence style, and if possible vertical lines should be
avoided.}\label{tbl-overview}\tabularnewline
\end{longtable}

\section{Illustrations}\label{sec-illustrations}

For a simple illustration of basic Poisson and NB count regression the
\texttt{quine} data from the \pkg{MASS} package is used. This provides
the number of \texttt{Days} that children were absent from school in
Australia in a particular year, along with several covariates that can
be employed as regressors. The data can be loaded by

\begin{verbatim}
R> data(mtcars)
\end{verbatim}

and a basic frequency distribution of the response variable is displayed
in \textbf{?@fig-quine}.

\begin{tcolorbox}[enhanced jigsaw, breakable, left=2mm, leftrule=.75mm, opacityback=0, arc=.35mm, rightrule=.15mm, bottomrule=.15mm, colback=white, toprule=.15mm]

For code input and output, the style files provide dedicated
environments. Either the ``agnostic'' \texttt{\{CodeInput\}} and
\texttt{\{CodeOutput\}} can be used or, equivalently, the environments
\texttt{\{Sinput\}} and \texttt{\{Soutput\}} as produced by \fct{Sweave}
or \pkg{knitr} when using the \texttt{render\_sweave()} hook. Please
make sure that all code is properly spaced, e.g., using
\texttt{y\ =\ a\ +\ b\ *\ x} and \emph{not} \texttt{y=a+b*x}. Moreover,
code input should use ``the usual'' command prompt in the respective
software system. For \proglang{R} code, the prompt
\texttt{R\textgreater{}} should be used with \texttt{+} as the
continuation prompt. Generally, comments within the code chunks should
be avoided -- and made in the regular {\LaTeX} text instead. Finally,
empty lines before and after code input/output should be avoided (see
above).

\end{tcolorbox}

\section{Summary and discussion}\label{sec-summary}

\begin{tcolorbox}[enhanced jigsaw, breakable, left=2mm, leftrule=.75mm, opacityback=0, arc=.35mm, rightrule=.15mm, bottomrule=.15mm, colback=white, toprule=.15mm]

As usual\ldots{}

\end{tcolorbox}

\section*{Computational details}\label{computational-details}
\addcontentsline{toc}{section}{Computational details}

\begin{tcolorbox}[enhanced jigsaw, breakable, left=2mm, leftrule=.75mm, opacityback=0, arc=.35mm, rightrule=.15mm, bottomrule=.15mm, colback=white, toprule=.15mm]

If necessary or useful, information about certain computational details
such as version numbers, operating systems, or compilers could be
included in an unnumbered section. Also, auxiliary packages (say, for
visualizations, maps, tables, \ldots) that are not cited in the main
text can be credited here.

\end{tcolorbox}

The results in this paper were obtained using
\proglang{R}\textasciitilde3.4.1 with the
\pkg{MASS}\textasciitilde7.3.47 package. \proglang{R} itself and all
packages used are available from the Comprehensive \proglang{R} Archive
Network (CRAN) at {[}https://CRAN.R-project.org/{]}.

\section*{Acknowledgments}\label{acknowledgments}
\addcontentsline{toc}{section}{Acknowledgments}

\begin{tcolorbox}[enhanced jigsaw, breakable, left=2mm, leftrule=.75mm, opacityback=0, arc=.35mm, rightrule=.15mm, bottomrule=.15mm, colback=white, toprule=.15mm]

All acknowledgments (note the AE spelling) should be collected in this
unnumbered section before the references. It may contain the usual
information about funding and feedback from colleagues/reviewers/etc.
Furthermore, information such as relative contributions of the authors
may be added here (if any).

\end{tcolorbox}

\section*{References}\label{references}
\addcontentsline{toc}{section}{References}

\renewcommand{\bibsection}{}
\bibliography{bibliography.bib}

\newpage{}

\section*{More technical details}\label{sec-techdetails}
\addcontentsline{toc}{section}{More technical details}

\begin{tcolorbox}[enhanced jigsaw, breakable, left=2mm, leftrule=.75mm, opacityback=0, arc=.35mm, rightrule=.15mm, bottomrule=.15mm, colback=white, toprule=.15mm]

Appendices can be included after the bibliography (with a page break).
Each section within the appendix should have a proper section title
(rather than just \emph{Appendix}).

For more technical style details, please check out JSS's style FAQ at
{[}https://www.jstatsoft.org/pages/view/style\#frequently-asked-questions{]}
which includes the following topics:

\begin{itemize}
\tightlist
\item
  Title vs.~sentence case.
\item
  Graphics formatting.
\item
  Naming conventions.
\item
  Turning JSS manuscripts into \proglang{R} package vignettes.
\item
  Trouble shooting.
\item
  Many other potentially helpful details\ldots{}
\end{itemize}

\end{tcolorbox}

\section*{Using BibTeX}\label{sec-bibtex}
\addcontentsline{toc}{section}{Using BibTeX}

\begin{tcolorbox}[enhanced jigsaw, breakable, left=2mm, leftrule=.75mm, opacityback=0, arc=.35mm, rightrule=.15mm, bottomrule=.15mm, colback=white, toprule=.15mm]

References need to be provided in a \textsc{Bib}{\TeX} file
(\texttt{.bib}). All references should be made with \texttt{@cite}
syntax. This commands yield different formats of author-year citations
and allow to include additional details (e.g.,pages, chapters, \dots) in
brackets. In case you are not familiar with these commands see the JSS
style FAQ for details.

Cleaning up \textsc{Bib}{\TeX} files is a somewhat tedious task --
especially when acquiring the entries automatically from mixed online
sources. However, it is important that informations are complete and
presented in a consistent style to avoid confusions. JSS requires the
following format.

\begin{itemize}
\tightlist
\item
  item JSS-specific markup (\texttt{\textbackslash{}proglang},
  \texttt{\textbackslash{}pkg}, \texttt{\textbackslash{}code}) should be
  used in the references.
\item
  item Titles should be in title case.
\item
  item Journal titles should not be abbreviated and in title case.
\item
  item DOIs should be included where available.
\item
  item Software should be properly cited as well. For \proglang{R}
  packages \texttt{citation("pkgname")} typically provides a good
  starting point.
\end{itemize}

\end{tcolorbox}




\end{document}
